\documentclass{pset}
\usepackage{jdh}

\renewcommand\maketitle{
  \vspace*{-5em}
  \begin{center}\framebox{\parbox{\boxlength}{\bf
    CS 4780, Fall 2013\hfill Project Proposal\hfill 22 October 2013
  }}
  \end{center}
  \vspace{1em}
}

\begin{document}

\maketitle

\section*{Motivation}

One's career is among the most central pieces of a person's life. While there
are many aspects to a job, such  as corporate culture, domain, etcetera, salary
 is an unavoidably important facet. To improve the job search experience and,
more importantly, bring transparency to the salaries behind each position, we
would like to estimate a position's salary from its provided posting. By doing
so, we will be helping the market two-fold: users will more easily be able to
filter down the immense number of postings already present to find only jobs
they are interested in with acceptable salaries to boot, while simultaneously
helping to move towards a more efficient labor market as employers are able to
more explicitly compete on pay.

From a technical standpoint, it's an interesting problem because we have a
large body of data to work with (100's of thousands of records) with fairly
little structure to each record. This allows us a wide variety of approaches,
both in how to parse the data and how to learn from it.

\section*{Problem Statement}

Our project's primary goal is to determine if, and how well, one can predict a
position's salary from a job posting alone. Similarly, we would also like to
determine which aspects of a job posting most strongly correlate with its
salary. In answering these questions, we also intend to determine which
strategy is best used to learn from such unstructured data. For example, is it
best to test on simple word presence? Perhaps by phrase? Will an SVM
significantly outperform a simple kNN model?

\section*{Approach}

The main thrust of our approach is a
\link{http://en.wikipedia.org/wiki/Boosting_(machine_learning)}{boosting}
algorithm that takes advantage of multiple classification algorithms. Using the
provided training and test datasets, we will use decision trees, SVMs, and a
naïve Bayes classifier to construct a super-classifier with greater predictive
power than any of its individual components.

{\it Decision Trees} - Borrowing from the concept of $n$-fold cross-validation,
we will train limited- height decision trees on various subsamples of the
training and test sets using the ID3 algorithm. Given a test instance and each
tree's classification of that instance, we will generate a weight distribution
across all possible classes for the instance, such that the weight of class $y$
is the proportion of trees that classify $x$ as $y$.

{\it SVM} - We will train a soft-margin multi-class SVM on the given training
set that, given a test instance $x$, outputs a distribution of weights across
all possible classifications of that instance.

{\it Naïve Bayes} - We will generate a naïve Bayes classifier from the provided
training set using the methods described in lecture. This classifier, given a
test instance $x$, will output a probability distribution across all possible
classes, which will serve as a weight distribution for boosting.

{\it Boosting} - We will implement a boosting algorithm that, given the weight
distributions from applying the decision tree, SVM, and naïve Bayes classifiers
to a test instance $x$, applies a weight to each of these distributions, sums
them together, and selects the class whose weighted sum is greatest.

\section*{Resources}

The majority of our resources will come directly from the Kaggle competition.
All data that we use will come from the provided datasets, available on the
\link{https://www.kaggle.com/c/job-salary-prediction/data}{competition website}.
This includes job ads in their full text as well as information about their
location and some specifics of the job in question.

Software we use will be SVM light, for our SVMs, and our own software written
in MATLAB, Python, or both for the decision trees and other potential
classifiers.

Our main research resource will likely be web searching to find up to date
information on salary predictors, so we can get a better understanding of how
to use the provided data efficiently. If we need to further research machine
learning techniques, the books for the course serve as a starting point.

\section*{Schedule}

We have created this schedule by incorporating particular milestones of the
development project process with the deadlines that were given to us in the
assignment specification:

{\bf$22^{nd}$ October:} Submit project proposal for feedback

{\bf$24^{th}$ October:} Peer reviews for project proposal due

{\bf$29^{th}$ October:} Finalize model parameters

{\bf$12^{th}$ November:} Initial coding of model complete

{\bf$19^{th}$ November:} Model mixing (boosting) code complete

{\bf$21^{st}$ November:} Submit progress report

{\bf$26^{th}$ November:} Finalized data and performance considerations

{\bf$30^{th}$ November:} Report draft done and begin revising

{\bf$03^{rd}$ December:} Poster complete

{\bf$05^{th}$ December:} Poster presentation.  Submit poster

{\bf$06^{th}$ December:} Peer reviews for posters due

{\bf$08^{th}$ December:} Project report complete, begin revising

{\bf$11^{th}$ December:} Final project report (and code) due

{\bf$18^{th}$ December:} Peer reviews for project reports due

\end{document}
